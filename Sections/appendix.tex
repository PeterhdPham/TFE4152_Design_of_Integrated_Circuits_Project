\section{Appendix}
\subsection{Aimspice}

\begin{lstlisting}
* --------------------------------------------------------------
* 2-Input NAND Gate
* Ports: A, B (Inputs), Out (Output), Vdd (Positive Supply), Vss (Ground)
* --------------------------------------------------------------
.subckt NAND A B Out Vdd Vss

    * PMOS transistors
    XMP1 Out A Vdd Vdd pmos1v w=P_Width l=P_Length 
    XMP2 Out B Vdd Vdd pmos1v w=P_Width l=P_Length 

    * NMOS transistors
    XMN1 Out A NS1 Vss nmos1v w=N_Width l=N_Length 
    XMN2 NS1 B Vss Vss nmos1v w=N_Width l=N_Length 

.ends

* --------------------------------------------------------------
* 3-Input NAND Gate
* Ports: InputA, InputB, InputC (Inputs), Output (Output), 
* Vdd (Positive Supply), Vss (Ground)
* --------------------------------------------------------------
.subckt NAND_3 InputA InputB InputC Output Vdd Vss

    * PMOS transistors
    XMP1 Output InputA Vdd Vdd pmos1v w=P_Width l=P_Length 
    XMP2 Output InputB Vdd Vdd pmos1v w=P_Width l=P_Length
    XMP3 Output InputC Vdd Vdd pmos1v w=P_Width l=P_Length

    * NMOS transistors
    XMN1 Output InputA N1 Vss nmos1v w=N_Width l=N_Length 
    XMN2 N1 InputB N2 Vss nmos1v w=N_Width l=N_Length
    XMN3 N2 InputC Vss Vss nmos1v w=N_Width l=N_Length  

.ends

* --------------------------------------------------------------
* DFF with asynchronous reset circuit
* Ports: CLK, InputData, Reset, Output, Vdd, Vss
* --------------------------------------------------------------
.subckt DFF CLK D Res Output NandOut3 Vdd Vss

    * 2-Input NAND Gate
    XN21 NandOut1 SetNode NandOut2 Vdd Vss NAND
    XN22 SetNode NandOut3 Output Vdd Vss NAND

    * 3-Input NAND Gate
    XN31 NandOut2 CLK Res SetNode Vdd Vss NAND_3 
    XN32 SetNode CLK NandOut1 ResetNode Vdd Vss NAND_3 
    XN33 ResetNode D Res NandOut1 Vdd Vss NAND_3 
    XN34 Output ResetNode Res NandOut3 Vdd Vss NAND_3 

.ends
\end{lstlisting}

\begin{lstlisting}
*inputScenarioW1.cir:

*Waveform: To Test Data Sampling and Clock Edge
vclk Clk 0 pulse (0 vdd_value 5n 0.1n 0.1n 10n 20n)     
vres Reset 0 pulse (0 vdd_value 0 0.1n 0.1n 39n 39n)       
vd  Data 0 pulse (0 vdd_value 10n 0.1n 0.1n 20n 40n)
.tran 0.0001n 200n 10n
\end{lstlisting}

\begin{lstlisting}
*inputScenarioW2.cir:

*Waveform: To Test the Datasampling when Data stays the same for a few Clock Edges
vclk Clk 0 pulse (0 vdd_value 5n 0.1n 0.1n 10n 20n)     
vres Reset 0 pulse (0 vdd_value 0 0.1n 0.1n 0n 0n)       
vd  Data 0 pulse (0 vdd_value 10n pRiseT pFallT 35n 80n)     
.tran 0.0001n 200n 10n
\end{lstlisting}

inputScenarioW3.cir:
\begin{lstlisting}
*Waveform: To Test the Synchronous Reset
vclk Clk 0 pulse (0 vdd_value 5n 0.1n 0.1n 10n 20n)     
vres Reset 0 pulse (vdd_value 0 5n 0.1n 0.1n 10n 40n)       
vd  Data 0 pulse (0 vdd_value 10n pRiseT pFallT 40n 40n)
.tran 0.0001n 200n 10n
\end{lstlisting}

\begin{lstlisting}
*inputScenarioS1.cir:
*Waveform: To test static leacage current when Clk:0, Data:0, Reset:1, Q:1
vclk Clk 0 pulse    (vdd_value  0    5n     0.1n    0.1n    5000n     5001n)     
vd  Data 0 pulse    (vdd_value  0    5n     0.1n    0.1n    5000n     5001n)   
vres Reset 0 pulse  (0  vdd_value    5n     0.1n    0.1n    5000n     5001n)   
.tran 0.01 9000n 5100n
\end{lstlisting}

\begin{lstlisting}
*inputScenarioS1.cir:
*Waveform: To test static leacage current when Clk:0, Data:1, Reset:1, Q:0
vclk Clk 0 pulse    (vdd_value  0    5n     0.1n    0.1n    5000n     5001n)     
vd  Data 0 pulse    (vdd_value  0    5n     0.1n    0.1n    5000n     5001n)   
vres Reset 0 pulse  (0  vdd_value    5n     0.1n    0.1n    5000n     5001n)   
.tran 0.01 9000n 5100n
\end{lstlisting}

\begin{lstlisting}
*inputScenarioS1.cir:
*Waveform: To test static leacage current when Clk:0, Data:1, Reset:1, Q:1
vclk Clk 0 pulse    (vdd_value  0    5n     0.1n    0.1n    5000n     5001n)     
vd  Data 0 pulse    (vdd_value  0    5n     0.1n    0.1n    5000n     5001n)   
vres Reset 0 pulse  (0  vdd_value    5n     0.1n    0.1n    5000n     5001n)   
.tran 0.01 9000n 5100n
\end{lstlisting}

\begin{lstlisting}
*inputScenarioS1.cir:
*Waveform: To test static leacage current when Clk:0, Data:0, Reset:1, Q:0
vclk Clk 0 pulse    (vdd_value  0    5n     0.1n    0.1n    5000n     5001n)     
vd  Data 0 pulse    (vdd_value  0    5n     0.1n    0.1n    5000n     5001n)   
vres Reset 0 pulse  (0  vdd_value    5n     0.1n    0.1n    5000n     5001n)   
.tran 0.01 9000n 5100n
\end{lstlisting}

\begin{lstlisting}
    * param.cir:
    
    * Scenarios:
    .include gpdk90nm_#1.cir 
    .include DFF.cir
    .include inputScenario#3
    .option temp= #5
    
    * Device global parameters:
    .param vdd_value = #4
    vdd 1 0 vdd_value
    
    * Device parameters for N-MOSFETs:
    .param N_Width= #6
    .param N_Length = #7
    
    * Device parameters for P-MOSFETs:
    .param P_Width= #8
    .param P_Length = #9
    \end{lstlisting}
    
    To streamline the the simulation process a bash script has been used to simulate all the scenarios and then saved the CSV file generated. The process began with running simulations for different VDD values below 1V while the MOSFET parameters were:
    
    \begin{lstlisting}
    * Device parameters for N-MOSFETs:
    .param N_Width= 0.1u
    .param N_Length = 0.1u
    
    * Device parameters for P-MOSFETs:
    .param P_Width= 0.3u
    .param P_Length = 0.1u
    \end{lstlisting}