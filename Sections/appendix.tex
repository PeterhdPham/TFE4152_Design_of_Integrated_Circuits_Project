\section{Appendix}
\subsection{Aimspice}
\begin{lstlisting}

    * Library and model files

                  
    * --------------------------------------------------------------
    * 2-Input NAND Gate
    * Ports: A, B (Inputs), Out (Output), Vdd (Positive Supply), Vss (Ground)
    * --------------------------------------------------------------
    .subckt NAND A B Out Vdd Vss
    
        * PMOS transistors
        XMP1 Out A Vdd Vdd pmos1v w=P_Width l=P_Length 
        XMP2 Out B Vdd Vdd pmos1v w=P_Width l=P_Length 
    
        * NMOS transistors
        XMN1 Out A NS1 Vss nmos1v w=N_Width l=N_Length 
        XMN2 NS1 B Vss Vss nmos1v w=N_Width l=N_Length 
    
    .ends
    
    * --------------------------------------------------------------
    * 3-Input NAND Gate
    * Ports: InputA, InputB, InputC (Inputs), Output (Output), 
    * Vdd (Positive Supply), Vss (Ground)
    * --------------------------------------------------------------
    .subckt NAND_3 InputA InputB InputC Output Vdd Vss
    
        * PMOS transistors
        XMP1 Output InputA Vdd Vdd pmos1v w=P_Width l=P_Length 
        XMP2 Output InputB Vdd Vdd pmos1v w=P_Width l=P_Length
        XMP3 Output InputC Vdd Vdd pmos1v w=P_Width l=P_Length
    
        * NMOS transistors
        XMN1 Output InputA N1 Vss nmos1v w=N_Width l=N_Length 
        XMN2 N1 InputB N2 Vss nmos1v w=N_Width l=N_Length
        XMN3 N2 InputC Vss Vss nmos1v w=N_Width l=N_Length  
    
    .ends
    
    

    
    * --------------------------------------------------------------
    xDFF Clk Data Reset Out NandOut3 1 0 DFF
    
    * Device parameters for N-MOSFETs
    .param N_Length = .3u
    .param N_Width= 0.3u 
    
    
    * Device parameters for P-MOSFETs
    .param P_Length = .3u
    .param P_Width= 0.3u 
    
    .param vdd_value = 0.6
    
    * Pulse signal parameters for input waveforms
    .param pDelayT= 0       
    .param pRiseT= 0.1n    
    .param pFallT= 0.1n     
    .param pPulseWidth= 10n
    .param pPeriod= 20n    
    
    * Power supply definition
    vdd 1 0 vdd_value        
    
    * DC voltage sources for test purposes
    vclk Clk 0 0               
    vd Data 0 vdd_value
    vres Reset 0 vdd_value
    .plot i(vdd)
    
    *Waveform 1: To Test Data Sampling and Clock Edge
    *vclk Clk 0 pulse (0 vdd_value 5n 0.1n 0.1n 10n 20n)     
    *vres Reset 0 pulse (0 vdd_value 0 0.1n 0.1n 39n 39n)       
    *vd  Data 0 pulse (0 vdd_value 10n pRiseT pFallT 20n 40n) 
    
    *Waveform 3: To Test the Data sampling during stable clock edges
    *vclk Clk 0 pulse (0 vdd_value 5n 0.1n 0.1n 10n 20n)     
    *vres Reset 0 pulse (0 vdd_value 0 0.1n 0.1n 0n 0n)       
    *vd  Data 0 pulse (0 vdd_value 10n pRiseT pFallT 35n 80n) 
    *.plot v(Data) v(Clk) v(Out)
    
    *Waveform 2: To Test the Asynchronous Reset
    *vclk Clk 0 pulse (0 vdd_value 5n 0.1n 0.1n 10n 20n)     
    *vres Reset 0 pulse (0 vdd_value 0 0.1n 0.1n 34n 38n)       
    *vd  Data 0 pulse (0 vdd_value 10n pRiseT pFallT 20n 40n)
    
    
    *Waveform 4: To Test the Synchronous Reset
    *vclk Clk 0 pulse (0 vdd_value 5n 0.1n 0.1n 10n 20n)     
    *vres Reset 0 pulse (vdd_value 0 5n 0.1n 0.1n 10n 40n)       
    *vd  Data 0 pulse (0 vdd_value 10n pRiseT pFallT 20n 40n)
    *.plot v(Data) v(Clk) v(Out) v(Reset)
    
    
    *****Simulation******
    .tran 0.00001n 200n 2n
    .include gpdk90nm_ss.cir
    .option temp=0
    
\end{lstlisting}
% \lstinputlisting[language=Python]{../figures/Problem1.py}

% \subsection*{Python Code for problem2}
% \lstinputlisting[language=Python]{../figures/Problem3.py}

% \subsection*{Python Code for problem5}
% \lstinputlisting[language=Python]{../figures/Problem4.py}
