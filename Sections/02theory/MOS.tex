\subsection{MOS}
MOS short for metal-oxide semiconductor are as of now, the most used technology in microcircuits are MOS transistors. These circuits consist primarily of n-channel (NMOS) and p-channel (PMOS) transistors, and when a microcircuit consist of both NMOS and PMOS transistors it's called a complementary MOS (CMOS).\cite[p. 14]{carusone_2012_analog}

\importimagewcaptionw{figures/Analog_Integrated_Circuit_Design/cross_section_NMOS.png}{Cross section of a typical NMOS\cite[p. 14]{carusone_2012_analog}}{0.8}


As seen in Fig. \ref{fig:figures/Analog_Integrated_Circuit_Design/cross_section_NMOS.png} MOS transistors consist mainly of $n^+$ source and drain region, sepparated by a $p^-$ substrate in the middle there is a gate made of Polysilicon. Depending on if whenever it is a PMOS or NMOS the transistors works as a switch where the gate voltage $V_G$ controlls current from the source region to drain region. 

The properties of a MOS transistor are mainly affected of distance between the source and drain region called channel length (L) and the width of the channel (W). The supply voltage does also affect the properties ot the transistor. 



\subsubsection{NMOS}
NMOS transistors works like a switch that are ``on'' when $V_G < V_{tn}$, where $V_{tn}$ is the transistor treshold voltage of the NMOS. As seen in Fig. \ref{fig:figures/Analog_Integrated_Circuit_Design/cross_section_NMOS.png} there is a thin layer of silicon dioxide ($\text{SiO}_2$) working as a insulator between the gate and the substrate. This makes the gate and the surface of the silicon behave like parallel plate capacitor. Whenever $V_G\ll0$ as seen in Fig. \ref{subfig:figures/Analog_Integrated_Circuit_Design/NMOS_off.png} the positive charge in the substrate wil be attracted to the channel region, forming the \textit{accumulated channel}. This is because the substrate is doped $p^-$ leading to the nagative voltage has increasing the channel doping to $p^+$. In this state the only current that flow between the source and drain is the leakage current.

In the case where $V_G\gg0$ as shown in Fig. \ref{subfig:figures/Analog_Integrated_Circuit_Design/NMOS_on.png}, the opposite happens. Now the positive voltage leads to an attraction of negative charge from the source and drain regions leading to a current flow between source and drain \cite[p. 16]{carusone_2012_analog}.

\imageontop{figures/Analog_Integrated_Circuit_Design/NMOS_off.png}{When $V_G\ll0$, there is an \textit{accumulated channel} resulting in no current flow(other than leakage current).}{figures/Analog_Integrated_Circuit_Design/NMOS_on.png}{When $V_G\gg0$, there is a channel present resulting in a current flow from drain to source}{NMOS transistor \cite[p. 17]{carusone_2012_analog}}{NMOS_transistor_on_off}{0.7}



\subsubsection{PMOS}
PMOS transistors in terms of functionality works opposite of the NMOS, where it is ``on'' when $V_G > V_{tp}$. This is because it uses p-type dopants in the gate region.\cite{a2023_definition}.

\subsubsection{Subthreshold leakage}
In terms of MOS fransistors, there are three leakage currents that we need to take into consideration: subthreshold leakage, gate leakage, and junction leakage. \cite[p. 42]{carusone_2012_analog} Especially subthreshold leakage. The MOS is in the subthreshold region when the \textit{effective gate-source voltage} ($V_{eff}$) \cite[p. 17]{carusone_2012_analog} given by
\begin{equation}
    V_{eff}\equiv V_{GS}-V_{tn}
    \label{eq:effective_gate-source_voltage}
\end{equation}

is negative. Here $V_{GS}$ is the gate-source voltage and $V_{tn}$ is the \textit{transistor threshold voltage}. In this subthreshold region there will be a drain current ($I_{D(sub-th)}$) given by

\begin{equation}
    I_{D(sub-th)}\cong I_{D0} \left(\frac{W}{L}\right) e^{(\frac{qV_{eff}}{NkT})}
    \label{eq:subthreshold_current}
\end{equation}

where

\begin{equation}
    n=\frac{C_{ox}+C_{j0}}{C_{ox}}\approx 1.5
    \label{eq:n}
\end{equation}

\begin{equation}
    I_{D0}=(n-1)\mu_nC_{ox}\left(\frac{kT}{q}\right)^2
\end{equation}

from equation \ref{eq:subthreshold_current} we can se that the current $I_{D(sub-th)}\neq 0$ when $V_{GS}=0V$. This is the so called \textit{subthreshold leakage} given by 

\begin{equation}
    \mathrm{I}_{\mathrm{off}}=\mathrm{I}_{\mathrm{DO}}\left(\frac{\mathrm{W}}{\mathrm{L}}\right) \mathrm{e}^{\left(-\mathrm{q} \mathrm{V}_{\mathrm{t}} / \mathrm{nkT}\right)}=(\mathrm{n}-1) \mu_{\mathrm{n}} \mathrm{C}_{\mathrm{ox}}\left(\frac{\mathrm{W}}{\mathrm{L}}\right)\left(\frac{\mathrm{kT}}{\mathrm{q}}\right)^2 \mathrm{e}^{\left(-\mathrm{qV} \mathrm{V}_{\mathrm{t}} / \mathrm{nkT}\right)}
    \label{eq:subthreshold_leakage}
\end{equation}

\subsubsection{MOS speeeeeeeeeeeeeeeeeeed}
Another thing that needs to be taken into consideration is the timings of the CMOS circuit. Which in this case is 
\begin{itemize}
    \item Propagnation delay low-high ($t_{pLH}$)
    \subitem 
    \item Propagnation delay high-low ($t_{pHL}$)
    \item Rise time ($t_r$) 
    \item Fall time ($t_f$)
\end{itemize}