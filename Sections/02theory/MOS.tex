\subsection{CMOS}
CMOS short for complementary metal-oxide semiconductor are circuits that consist of both NMOS and PMOS transistors.\cite[p. 14]{carusone_2012_analog}. This technology is popular as they are both fast and consume low power. As low static leakage power is often wanted, a technique to minimize this is to use static CMOS. This technique utilizes the PMOS as pull-up (PUN) and NMOS as pull-down (PDN), this the output will be directly connected to either the $V_{DD}$ or ground with an exeption of when it is switching. This way there will never be a direct path from $V_{DD}$ to ground \cite{a2018}. From figure \ref{fig:static_cmos_nand} we can see an implementation of 2-input NAND and 3-Input NAND that are a typical example of a logic gate that usilises Static CMOS.
\imagesidebyside{figures/drawio/Transistor/transisotr_nand2.png}{2 input NAND}{figures/drawio/Transistor/transistor_nand3.png}{3 input NAND}{Implementation of 2- and 3- input NANDS using static CMOS}{static_cmos_nand}{0.4}


It is worth noticing that NMOS are faster than PMOS ``(almost twice as fast) and the ON-resistance of NMOS being half of PMOS even with the same geometry and operation condition''\cite{kusumitha_2019_why}. MOSFET technology between $0.18\mu m$ and $0.45\mu m$ have NMOS mobility of $\mu_nC_{ox}\approx 270-280 \frac{\mu A}{V^2}$ and PMOS mobility of $\mu_pC_{ox}\approx 70 \frac{\mu A}{V^2}$ \cite{carusone_2012_analog}. The relation between the channel dimmensions and the mobility og the MOSFETS

\begin{align}
    \mu_nC_{ox}\frac{W_n}{L_n}&=\mu_pC_{ox}\frac{W_p}{L_p}\\
    \beta_n&=\beta_p
    \label{eq:beta_ratio}
\end{align}

is desirable when making CMOS circuits as it maximizes noise margins, making it less prone to the output signal being corrupted. Here $W_n$ and $L_n$ is the width and lenght if the NMOS, while $W_n$ and $L_n$ is the width and lenght if the PMOS. As the MOSFET length ($L$) are usally the chosen as same for both transistors, the desired behavior can de described as the beta ratio $r$ \cite[p. 90]{neilheweste_2015_cmos}

\begin{equation}
    r=\frac{\mu_n W_n}{\mu_p W_p}=1
    \label{eq:betaratio}
\end{equation}

\subsubsection{Subthreshold leakage}
In terms of PMOS and NMOS transistors, there are three leakage currents that should be taken into consideration: subthreshold leakage, gate leakage, and junction leakage. \cite[p. 42]{carusone_2012_analog} Especially subthreshold leakage. The MOSFET is in the subthreshold region when the \textit{effective gate-source voltage} ($V_{eff}$) \cite[p. 17]{carusone_2012_analog} given by
\begin{equation}
    V_{eff}\equiv V_{GS}-V_{tn}
    \label{eq:effective_gate-source_voltage}
\end{equation}

is negative. Here $V_{GS}$ is the gate-source voltage and $V_{tn}$ is the \textit{transistor threshold voltage}. In this subthreshold region there will be a drain current ($I_{D(sub-th)}$) given by

\begin{equation}
    I_{D(sub-th)}\cong I_{D0} \left(\frac{W}{L}\right) e^{(\frac{qV_{eff}}{NkT})}
    \label{eq:subthreshold_current}
\end{equation}

where

\begin{equation}
    n=\frac{C_{ox}+C_{j0}}{C_{ox}}\approx 1.5
    \label{eq:n}
\end{equation}

\begin{equation}
    I_{D0}=(n-1)\mu_nC_{ox}\left(\frac{kT}{q}\right)^2
\end{equation}

from equation \ref{eq:subthreshold_current} the current $I_{D(sub-th)}\neq 0$ when $V_{GS}=0V$. This is the so called \textit{subthreshold leakage} given by 

\begin{equation}
    \mathrm{I}_{\mathrm{off}}=\mathrm{I}_{\mathrm{DO}}\left(\frac{\mathrm{W}}{\mathrm{L}}\right) \mathrm{e}^{\left(-\mathrm{q} \mathrm{V}_{\mathrm{t}} / \mathrm{nkT}\right)}=(\mathrm{n}-1) \mu_{\mathrm{n}} \mathrm{C}_{\mathrm{ox}}\left(\frac{\mathrm{W}}{\mathrm{L}}\right)\left(\frac{\mathrm{kT}}{\mathrm{q}}\right)^2 \mathrm{e}^{\left(-\mathrm{q} \mathrm{V}_{\mathrm{t}} / \mathrm{nkT}\right)}
    \label{eq:subthreshold_leakage}
\end{equation}

This gives us the static leakage power \cite{departmentofelectronicsystemsntnu_2023_dc}

\begin{equation}
    P_{stat}=I_{off}\cdot V_{DD}
    \label{eq:leakage_power}
\end{equation}

\subsubsection{Delays}
Another thing that needs to be taken into consideration is the timings of the CMOS circuit. Which in this case is (equation below are with respect to an inverter using Static CMOS) \cite{departmentofelectronicsystemsntnu_2023_delay}
\begin{itemize}
    \item \textbf{Propagnation delay low-high} ($t_{pLH}$)
    \subitem This is the delay between when the input signal is at 50\% of its destination value and when the output signal rises from low to 50\% of its high value.
    \item \textbf{Propagnation delay high-low} ($t_{pHL}$)
    \subitem This is the delay between when the input signal is at 50\% of its destination value value and when the output signal falls from high to 50\% of its low value.\cite{wikipediacontributors_2023_signal}
    \item \textbf{Rise time} $\left(t_r\right)$
    \subitem This is the time when the output signal rises from 10\% to 90\% of its max value.
    \item \textbf{Fall time} $\left(t_f\right)$
    \item \subitem This is the time when the output signal falls from 90\% to 10\% of its max value.
\end{itemize}


\importimagewcaptionh{figures/FromWeb/propagation_rise_fall.png}{The propagation delays, rise and falltimes of a NMOS transistor\cite{a2014_fig}}{0.4}

The gate and surface of the silicon behaves like parallel plate capasitor, the gate capacitance $C_g$ is given by \cite[p. 18]{carusone_2012_analog}

\begin{equation}
    C_g=WLC_{ox}
    \label{eq:gate_capacitance}
\end{equation}

where $C_{ox}$ is the gate capacitance per unit area. This is important as a bigger capacitans means longer charge up and charge down time. This combined with the parasitic resistance of the MOSFET given by \cite{departmentofelectronicsystemsntnu_2023_delay}

\begin{equation}
    \begin{aligned}
    & R_n=\frac{1}{\beta_n\left(V_{G S^{-}} V_{t n}\right)} \\
    & R_p=\frac{1}{\beta_p\left(V_{S G}-\left|V_{t p}\right|\right)}
    \end{aligned}
    \label{eq:parasitic_resistance}
\end{equation}

Where $R_p$ is the parasitic resistance of the PMOS, $R_n$ is the parasitic resistance of the NMOS. This gives us the rise time

\begin{equation}
    t_r=\tau_p \ln \left(\frac{V_{DD}}{0.9V_{DD}}\right)-\tau_p \ln \left(\frac{V_{DD}}{0.1V_{DD}}\right)
    \label{eq:risetime}
\end{equation}

and fall time

\begin{equation}
    t_f=\tau_n \ln \left(\frac{V_{DD}}{0.9V_{DD}}\right)-\tau_n \ln \left(\frac{V_{DD}}{0.1V_{DD}}\right)
    \label{eq:falltime}
\end{equation}

where $\tau_p=R_pC_{out}$ and $\tau_n=R_nC_{out}$, $C_{out}$ being all the capacitances. \cite{departmentofelectronicsystemsntnu_2023_delay}

\subsubsection{Process corners}
When applying an integrated circuit (IC) design to a silicon wafer, it is important to simulate processcorners that represetns the extreme scenarios of the parameters in the circuit. This is because a circuit utilising these ICs may run slower or faster then specified. To ensure that the designed ID will handle the extreme scenarios,  front end of line (FEOL) process corners are used to stresstest the IC in the scematic domain. There are 5 different corners: typical-typical (TT), fast-fast (FF), slow-slow (SS), fast-slow (FS), and slow-fast (SF) where the first letter refer to the NMOS corner and the second letter refers to the PMOS corner. These corners will simulate how the IC behaves at different switchtime speeds.\cite{wikipediacontributors_2022_process}