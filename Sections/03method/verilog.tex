\subsection{Verilog VHDL}


We need to assemble the circuits: the combinational circuits in the FSM, and the arithmetic circuits within the MAC unit. Additionally, we must implement an 8-bit register. Using Verilog, we'll ensure that all components operate in sync as intended.

\subsection{ADDERS}
A natural start making a MAC-unit, is an adder. They are used to perform addition, which is a basic arithmetic operation. Where adding is a escential part of its job.
\subsubsection{HALF-ADDER}
A simple adder with two 1-bits inputs and a 1-bit sum and carry output, is known as a \textbf{Half adder}. By making a Truthtable for the half-adder we can get the bolean expresion and make a circuit out of it.- refferer til DATDIG, og chatgbt dette
\begin{table}[H]
    \caption{Truthtable for Half Adder}                 
    \centering
    \begin{tabular}{|c|c|c|c|}
    \hline
    \(A\) & \(B\) & \(Sum\) & \(Carry\) \\
    \hline
    0 & 0 & 0 & 0 \\
    0 & 1 & 1 & 0 \\
    1 & 0 & 1 & 0 \\
    1 & 1 & 0 & 1 \\
    \hline
\end{tabular}
\end{table}
From the Truthtable we can this expresion:
\begin{equation}
    \begin{aligned}
        Sum&=\overline{A}B+A\overline{B}\\
        \Rightarrow&A\oplus B\\
        Carry&=AB\\ 
    \end{aligned}
\end{equation}
\begin{lstlisting}
    //{module {HALF_ADDER}}	 
`timescale 1 ns / 1 ps
module HALF_ADDER ( A ,B ,SUM ,CARRY );

input A ;
wire A ;
input B ;
wire B ;
output SUM ;
wire SUM ;
output CARRY ;
wire CARRY ;


xor(SUM, A, B);
and(CARRY, A, B);

endmodule
\end{lstlisting}

\begin{lstlisting}
    module TEST_HALF_ADDER;
	reg a;
	reg b;
	wire sum;
	wire carry;
	
	HALF_ADDER TESTHA( .A(a) ,.B(b) ,.SUM(sum) ,.CARRY(carry ));
	initial begin
		a=1'b0;
		forever #30	a=~a;
	end	
	initial begin
		b=1'b0;
		forever #15	b=~b;
	end	
	
	initial begin
		#60 $finish;
	end	  
endmodule
\end{lstlisting}


\importimagewcaption{verilog/Verilog-Half-adder.png}{}
\importimagewcaptionw{verilog/Half-adderer.png}{}{1}



\subsection{Adderer}
As the Half-adder get handle the lower order bit, we need a full adder wich get handle the carry. 
\begin{table}[H]
    \centering
    \begin{tabular}{|c|c|c|c|c|}
        \hline
        \(A\) & \(B\) & \(C_{\text{in}}\) & \(S\) & \(C_{\text{out}}\) \\
        \hline
        0 & 0 & 0 & 0 & 0 \\
        0 & 1 & 0 & 1 & 0 \\
        1 & 0 & 0 & 1 & 0 \\
        1 & 1 & 0 & 0 & 1 \\
        0 & 0 & 1 & 1 & 0 \\
        0 & 1 & 1 & 0 & 1 \\
        1 & 0 & 1 & 0 & 1 \\
        1 & 1 & 1 & 1 & 1 \\
        \hline
    \end{tabular}
    \caption{Sannhetstabell for Full Adder}
\end{table}
From the truth table we get this bolean expresion
\begin{equation}
    \begin{aligned}
        Sum&=\overline{A}B\overline{C}+A\overline{B}\overline{C_{\text{in}}}+\overline{A}\overline{B}C_{\text{in}}+ABC_{\text{in}}\\
        \Rightarrow&A\oplus B\oplus C_{\text{in}}\\
        C_{\text{out}}&=AB\overline{C_{\text{in}}}+\overline{A}BC_{\text{in}}+A\overline{B}C_{\text{in}}+ABC_{\text{in}}\\ 
        \Rightarrow&AB+C_{\text{in}}(A\oplus B)\\
    \end{aligned}
\end{equation}

\importimagewcaption{verilog/Verilog-Adder.png}{}
\importimagewcaptionw{verilog/Test-Adder.png}{}{1}
    
    \subsection{Mulitplikator}
    \importimagewcaption{verilog/Verilog-Multiplikator.png}{}
    \importimagewcaption{verilog/Test-Mulitplikator.png}{}
\begin{table}[H]
    \centering
    \begin{tabular}{|c|c|c|c|c|c|c|c|}
    \hline
    \(A_1\) & \(A_0\) & \(B_1\) & \(B_0\) & \(C_3\) & \(C_2\) & \(C_1\) & \(C_0\) \\
    \hline
    0 & 0 & 0 & 0 & 0 & 0 & 0 & 0 \\
    0 & 0 & 0 & 1 & 0 & 0 & 0 & 0 \\
    0 & 0 & 1 & 0 & 0 & 0 & 0 & 0 \\
    0 & 0 & 1 & 1 & 0 & 0 & 0 & 0 \\
    0 & 1 & 0 & 0 & 0 & 0 & 0 & 0 \\
    0 & 1 & 0 & 1 & 0 & 0 & 0 & 1 \\
    0 & 1 & 1 & 0 & 0 & 0 & 1 & 0 \\
    0 & 1 & 1 & 1 & 0 & 0 & 1 & 1 \\
    
    1 & 0 & 0 & 0 & 0 & 0 & 0 & 0 \\
    1 & 0 & 0 & 1 & 0 & 0 & 1 & 0 \\
    1 & 0 & 1 & 0 & 0 & 1 & 0 & 0 \\
    1 & 0 & 1 & 1 & 0 & 1 & 1 & 0 \\
    1 & 1 & 0 & 0 & 0 & 0 & 1 & 0 \\
    1 & 1 & 0 & 1 & 0 & 1 & 0 & 1 \\
    1 & 1 & 1 & 0 & 0 & 1 & 1 & 0 \\
    1 & 1 & 1 & 1 & 1 & 0 & 0 & 1 \\
    \hline
    \end{tabular}
    \caption{Sannhetstabell for 2-bit multiplikator}
    \end{table}

    \subsection{D-flip flop with asychron reset}
    \importimagewcaption{verilog/Verilog-D-flipflop.png}{}
    \importimagewcaption{verilog/D-flipflop.png}{}
    \begin{table}[H]
        \centering
        \begin{tabular}{|c|c|c|c|c|}
            \hline
            \(D\) & \text{CLK (edge)} & \(R\) & \(Q \text{ (next)}\) & \(\overline{Q} \text{ (next)}\) \\
            \hline
            x & \(\downarrow\) or no change & 1 & \(Q \text{ (prev)}\) & \(\overline{Q} \text{ (prev)}\) \\
            x & \(\uparrow\) & 1 & \(D\) & \(\overline{D}\) \\
            x & x & 0 & 0 & 1 \\
            \hline
        \end{tabular}
    \end{table}