\subsection{Method for FSM}
In designing our FSM, we used a structured approach inspired by  \cite{allaboutelectronics_2023_sequence}:
\begin{enumerate}
    \item Define the purpose of the machine
    \item Draw a state diagram
    \item Create a state table
    \item Select type of Flip-Flop
    \item Find the Boolean Functions for each Flip-Flop
    \item Draw a logic circuit
\end{enumerate}
For \textbf{defining the purpose of the purpose} we have to considered the project's specific criteria and the functional necessities of our MAC unit. The key criteria include:
\begin{itemize}
    \item The FSM has three inputs($CLK$, $RESET$, $RUN$).
    \begin{enumerate}
        \item $RESET$: A binary signal indicating when the circuit should reset
        \item $CLK$: A clock signal oscillating between 0 and 1.
        \item $RUN$: A control signal dictating whether to perform MAC operations or pause.
    \end{enumerate}
    \item The FSM should execute a sequence of three consecutive MAC operations followed by a pause for one clock cycle. This 3+1 pattern repeats continuously, with exceptions for $RUN$ and $RESET$ conditions.
    \begin{enumerate}
        \item If $Run$=0, no MAC operation should be run, and resume from the interrupted point when 
        $RUN$ is set to 1 again
        \item If Reset=1, , the MAC sequence resets, restarting with three operations followed by a pause when $RESET$ return to 0
    \end{enumerate}
    \item The FSM’s state should only be updated at the positive edge of the $CLK$ signal.
\end{itemize}

Our MAC unit updates with the rising edge of the $CLK$ input to our 8-bit register. The plan is to generate a signal indicating the state where output $PAUSE$=1.

For this we designed a \textbf{Statediagram} illustrated in Figure \ref{fig:figures/drawio/Digital/State_diagram.png}

\importimagewcaptionw{figures/drawio/Digital/State_diagram.png}{State diagram with PAUSE output}{0.6}

The \textbf{State table}, table \ref{State Table FSM}, is based on the statediagram. In Table \ref{State Table FSM}, $P_{0}$, $P_{1}$ represent the current states (Present State 0 and 1),  and $N_{0}$, $N_{1}$ are represent the next states (Next State 0 and 1). The inputs are $RESET$ and $RUN$, and the output is $PAUSE$

\begin{table}[H]
    \centering
    \caption{State Table FSM}
    \label{State Table FSM}
    \begin{tabular}{|c|c|c|c|c|c|c|}
        \hline
        $P_{0}$ & $P_{1}$ & RESET & RUN & $N_{0}$ & $N_{1}$ & PAUSE \\
        \hline
        x    & x    & 1     & x   & 1    & 1    & 1     \\
        0    & 0    & 0     & 0   & 0    & 0    & 0     \\
        0    & 0    & 0     & 1   & 1    & 0    & 0     \\
        1    & 0    & 0     & 0   & 1    & 0    & 0     \\
        1    & 0    & 0     & 1   & 0    & 1    & 0     \\
        0    & 1    & 0     & 0   & 0    & 1    & 0     \\
        0    & 1    & 0     & 1   & 1    & 1    & 0     \\
        1    & 1    & 0     & 0   & 1    & 1    & 1     \\
        1    & 1    & 0     & 1   & 0    & 0    & 1     \\
        \hline
        
    \end{tabular}
    \end{table}



The state diagram shown in figure..., have four states, and are using two flip flops to represent these states.Given that we have previously implemented D-flip flops with asynchronous reset (nevnt i 8-bits register), we adapted this design for our current needs. Notably, in this implementation, we are not using the reset input. This approach results in a circuit similiar to the one with the asynchronous reset, but without a reset input, shown in figure..

figure av register

From the Table \ref{State Table FSM} we can find the \textbf{boolean expresion for each Flip-Flop input}, wich are \emph{Next States} values, $N_{0}$ and $N_{1}$. 

\begin{equation} 
    \label{State equation}
    \begin{aligned}
    N_{0}&= \overline{P_{1}} \, \overline{P_{0}} \, \text{RUN} \, \overline{\text{RESET}} + \overline{P_{1}} \, P_{0} \, \overline{\text{RUN}} \, \overline{\text{RESET}} + P_{1} \, \overline{P_{0}} \, \text{RUN} \, \overline{\text{RESET}} + P_{1} \, P_{0} \, \overline{\text{RUN}} \, \overline{\text{RESET}} + \text{RESET}\\
    \Rightarrow&P_{0}\oplus RUN+ RESET\\
    N_{1}&= \overline{P_{1}} P_{0} \text{RUN} \overline{\text{RESET}} + P_{1} \overline{P_{0}} \overline{\text{RUN}} \overline{\text{RESET}} + P_{1} \overline{P_{0}} \text{RUN} \overline{\text{RESET}} + P_{1} P_{0} \overline{\text{RUN}} \overline{\text{RESET}+ RESET}\\
    \Rightarrow&\overline{P_{1}} P_{0} \text{RUN} + P_{1}(\overline{P_{0}} + \overline{\text{RUN}}) + \text{RESET}\\
    \text{PAUSE}&= P_{1} P_{0} \overline{\text{RUN}} \, \overline{\text{RESET}} + P_{1} P_{0} \text{RUN} \, \overline{\text{RESET}} + \text{RESET}\\
    \Rightarrow&P_{1} P_{0}+\text{RESET}\\
    \end{aligned}
\end{equation}

From the equations we can \ref{State equation} design a circuit with an $PAUSE$ output that is 1 when it should pause,  see figure ....


\importimagewcaption{figures/drawio/RTL/RTL_FSM_pause.png}{FSM with PAUSE output}

As mentioned we have to understand how our MAC works- the MAC is updated only by the reset signal and the CLK input to our 8-bit register. We have to modify our Finite State Machine (FSM) with these requirements, and make it to output two specific signals: $Clk_{OUT}$ and $RESET_{OUT}$. 

To generate a $Clk_{OUT}$ signal, we have to design a circuit using an AND gate with three inputs: $CLK$, $RUN$, $\overline{PAUSE}$. This configuration  ensure that we get a  modified clock singal that follows the 3+1 pattern. By combining this with the $RUN$ signal, we ensure that updates to the 8-bit register occur only as it supposed to be. It is important to note that $\overline{PAUSE}$ can still be active even when $RUN$ is deactivated (0), as it primarily controls the 3+1 operation pattern of the MAC. To synch the $RUN$ with the $CLK$ we also have made an D-flip flop. The $RUN$ in the equations in \ref{State equation}, is now the $Q$ output from this D-flip flop, and the $\overline{Q}$ output is $\overline{\text{RUN}}$  .


The $RESET_{OUT}$ is desigend by using the same flip-flop configuration as used in the state circuit. With this approach $RESET_{OUT}$, represented by the $\overline{Q}$ output from the Flip-Flops, is updated only on the rising clock edge. At the same time the $RESET$ in the equations in \ref{State equation}, is now the $Q$ output from this flip-flop. This setup ensures that the $PAUSE$ output does not update except on the rising edge of the clock.

The rescult for this FSM is shown in Figure \ref{fig:figures/drawio/RTL/RTL_FSM_no_delay.png}

\importimagewcaption{figures/drawio/RTL/RTL_FSM_no_delay.png}{FSM with two output: $RESET_{OUT$}, $CLK_{OUT$}}


Figure av hele FSM
    


        
    
    

