\subsection{AIMSpice}
In the AIMSpice part of this project the 1-bit version of the register from sec \ref{sec:Method for Accumulater} i.e. a single DFF with reset from figure \ref{figures/drawio/Logic/LOGIC_DFF_with_reset.png} have been simulated, with a desire for minimizing the static leakage power consumption while keeping the functionality for all the corners mentioned in sec \ref{sec:Process corners} and the temperatures $0^o C$, $27^o C$ and $70^o C$. The timings mentioned in sec \ref{sec:Delays} and maximizing of noise margins explained in sec \ref{sec:CMOS} were also taken into consideration when 

The timings mentioned in sec \ref{sec:Delays} as we can see from the equations \ref{eq:subthreshold_leakage}, \ref{eq:gate_capacitance} and \ref{eq:parasitic_resistance}, a change in the parameters $V_{DD}$ and $\frac{W}{L}$ both affect the static power consumption and the timing of the circuit. This way we want to find the sweetspot where we keep a low power without compensating to much on the timings.

\subsubsection{AIMSpice-implementation}
The DFF were implemented using gpdk 90nm technology, where the implementation at a transistor level can be seen in Fig. \ref{fig:figures/drawio/Transistor/transistor_circuit_DFF.png} First the logic gates were implemented as subcircuits

\importimagewcaptionh{figures/drawio/Transistor/transistor_circuit_DFF.png}{The implemented DFF at a transistor level}{0.9}

The fuctionality were simulated using three different wave pulses to show how the DFF samples alternating data at rising edge, how it samples data when data stays the same for a few clock cycles and lastly the reset functionality:


As shown in table \ref{tab:static_states} we can se that the DFF may be at 12 different static states (output and reset can't both be 0). Given $3+1$ pattern specified in the task in addition to the option of pause we know that the states where Clk=0 will occour more. With the assumption that the active low Reset state will be high more often than low. We end up with 4 states that Well look at when computing the static leakage power. These are the highlighted states in table \ref{tab:static_states}.

\begin{table}[H]
    \centering
    \caption{Static states of the DFF}
    \label{tab:static_states}
    \begin{tabular}{cccc}
        \hline
        \multicolumn{3}{c}{Input}                                 & Output                          \\ \hline
        Clk                             & Data       & Reset      & Q                               \\ \hline
        0                               & 0          & 0          & 0                               \\
        0                               & 1          & 0          & 0                               \\
        1                               & 0          & 0          & 0                               \\
        1                               & 1          & 0          & 0                               \\ \hline
        \multicolumn{1}{|c}{\textbf{0}} & \textbf{0} & \textbf{1} & \multicolumn{1}{c|}{\textbf{0}} \\ \hline
        \multicolumn{1}{|c}{\textbf{0}} & \textbf{1} & \textbf{1} & \multicolumn{1}{c|}{\textbf{0}} \\ \hline
        1                               & 0          & 1          & 0                               \\
        1                               & 1          & 1          & 0                               \\ \hline
        \multicolumn{1}{|c}{\textbf{0}} & \textbf{0} & \textbf{1} & \multicolumn{1}{c|}{\textbf{1}} \\ \hline
        \multicolumn{1}{|c}{\textbf{0}} & \textbf{1} & \textbf{1} & \multicolumn{1}{c|}{\textbf{1}} \\ \hline
        1                               & 0          & 1          & 1                               \\
        1                               & 1          & 1          & 1                              
    \end{tabular}
\end{table}

The mobility of the NMOS and PMOS are diffrent, therfore the subcircuit were implemented with an option to edit the parameters for both $\frac{W_N}{L_N}$ and $\frac{W_P}{L_P}$. $\#1-\#9$ are switched out with different parameter values to simulate the spesific physical parameters at different corners, temperatures and wave scenarios.

The increased PMOS channel Width is to compensate for the lower mobility of the PMOS given the equation \ref{eq:parasitic_resistance} in terms of delays. After simulating the for different VDD values, we simulated for different values for device parameters. After all the simulations the CSV files were used to plot the funtionality and computising the delays and the static leakage powers at the different states and scenarios. 