\subsubsection{Method for Multiplier} \label{Method-Multiplier}
As discussed in section \ref{sec:Multiplier} and illustrated in Figure (refferer til multiplier kretsen), our circuit desing is based on this model . With two 2-bits input, $A$ and $B$, and a 4-bit output $C$. To test and simulate this we used the program $Active-HDL Student Edition$ as a simulation tool, and programed in verilog. 

Legg til figur av multiplieren(med navn) PETER!


Figure illustrates a multiplier with the output denoted as $C[0:3]$. $C[3]$ is assigned as the least significant bit (LSB) and $C[0]$ as the most significant bit (MSB). Similiar with the input $A[0:1]$ and $B[0:1]$, where $A[0] $ and $B[0]$ are MSb, and $A[1]$ and $B[0]$ are LSb. In other way, we are using the number scheme \emph{MSb 0} \cite{wikipediacontributors_2023_bit}, in this project.

\begin{lstlisting}
    //{module {Multiplier}}	
`timescale 1 ns / 1 ps
module Multiplier ( A, B, C );
input [0:1] A;
input [0:1] B;
output [0:3] C;

and(C[3], A[1], B[1]);
and(OUT_1, A[0], B[0]);
and(OUT_2, A[0], B[1]);
and(OUT_3, A[1], B[0]);
and(OUT_4, OUT_2, OUT_3);
xor(C[2], OUT_2, OUT_3); 
xor(C[1], OUT_4, OUT_1);
and(C[0], OUT_1, OUT_4);
endmodule
\end{lstlisting}
