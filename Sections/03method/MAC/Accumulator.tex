\subsubsection{Method for Accumulater}

The \textbf{D-flip flop} (DFF) circuit is based on figure ...ref.. in section.... By assembling eight D-flip flop with eight data inputs $b[0:7]$ and corresponding $q[0:7]$ outputs, we make an \textbf{(8-bit register)}. The registers are connected in parallell by having shared $RESET$ and $CLK$ input. 

\imagesidebyside{figures/drawio/Logic/LOGIC_DFF.png}{Without reset input}{figures/drawio/Logic/LOGIC_DFF_with_reset.png}{With reset input}{Diagram of a DFF}{DFF}{0.49}
D-flip flop with asychron reset designed in verilog:
\begin{lstlisting}
module Dflipflop( CLK ,RESET ,DATA ,Q);
input CLK ;
input RESET ;
input DATA ;
output Q ;

nand N1(OUT_1, RESET,DATA, OUT_2);
nand N2(OUT_2, CLK, OUT_3, OUT_1);
nand N3(OUT_3, CLK, RESET, OUT_4);
nand N4(OUT_4, OUT_3, OUT_1);
nand N5(NQ, Q, RESET, OUT_2);
nand N6(Q, OUT_3, NQ);
endmodule
\end{lstlisting}
And as described the 8-bit register will look like this:

\begin{lstlisting}
module EIGHT_B_REG( b ,CLK_Out ,RESET_Out ,q );
input [0:7] b ;
input CLK_Out ;
input RESET_Out ;
output [0:7] q ;

Dflipflop u0 (	.CLK(CLK_Out) ,.RESET(RESET_Out) ,.DATA(b[0]) ,.Q(q[0])); 
Dflipflop u1 ( .CLK(CLK_Out) ,.RESET(RESET_Out) ,.DATA(b[1]) ,.Q(q[1]));
Dflipflop u2 ( .CLK(CLK_Out) ,.RESET(RESET_Out) ,.DATA(b[2]) ,.Q(q[2]));
Dflipflop u3 ( .CLK(CLK_Out) ,.RESET(RESET_Out) ,.DATA(b[3]) ,.Q(q[3]));
Dflipflop u4 ( .CLK(CLK_Out) ,.RESET(RESET_Out) ,.DATA(b[4]) ,.Q(q[4]));
Dflipflop u5 ( .CLK(CLK_Out) ,.RESET(RESET_Out) ,.DATA(b[5]) ,.Q(q[5]));
Dflipflop u6 ( .CLK(CLK_Out) ,.RESET(RESET_Out) ,.DATA(b[6]) ,.Q(q[6]));
Dflipflop u7 ( .CLK(CLK_Out) ,.RESET(RESET_Out) ,.DATA(b[7]) ,.Q(q[7]));
endmodule
\end{lstlisting}

The \textbf{Half Adder} and the \textbf{Full Adder} was designed on verilog, based on \ref{sec:ADDERS}, with respectively figures \ref{fig:figures/drawio/Logic/LOGIC_HALFADDER.png} and \ref{fig:figures/drawio/Logic/LOGIC_FULLADDER.png}.
\begin{lstlisting}
module HALF_ADDER ( A ,B ,SUM ,CARRY );
input A ;
input B ;
output SUM ;
output CARRY ;

xor(SUM, A, B);
and(CARRY, A, B);
endmodule
\end{lstlisting}
\begin{lstlisting}
module FULL_ADDER ( C_in ,A ,B ,SUM ,C_out );
input C_in ;
input A ;
input B ;
output SUM ;
output C_out ;

xor(OUT_1, A, B);
xor(SUM, C_in, OUT_1);
and(OUT_2, A, B);
and(OUT_3, OUT_1, C_in);
or(C_out, OUT_2, OUT_3);
endmodule
\end{lstlisting}

Adding the 4-bit output ($C[0:3])$ from the 2x2 Multiplier with the 8-bit output($q[0:7]$) requires to assemble half adder and full adder in a series. This arrangement ensures that the $SUM$ output of each adder stage contributes to the updated value of the register. As mentioned the output from the Multiplier is labeled as $C$, with $C[3]$ assigned as LSb and $C[0]$ as MSb. Conversely, the output from the register, assigns $q[7]$ as the LSb and $q[0]$ as the MSb.

\importimagewcaptionw{figures/drawio/Digital/adder_equation.png}{4-bit added with 8-bit equation}{0.6}

Figure \ref{fig:figures/drawio/Digital/adder_equation.png} illustrates that LSb for both $C$ and $q$ there is abscence of a carry input, therefor only necessesary with a half-adder with two inputs for q[7] and C[3]. For the next three following bits, the requirement for three inputs emerges. Therefore, a full adder is employed for each of the following stages, adding a bit from $C[2]$ to $C[0]$ and the corresponding bits from $q[6]$ to $q[4]$, along with the carry from the previous adder stage. With no more remaining bits from $C$ after the fourth bit, we have absence of an input in the full adder, therefor it will only be a necessity with the use of a half adders with inputs of the remainig q-bits and carry from the previous adder stage. MSB $q[0]$ does not require a dual-output, as there no no more half adders to transfer a carry bit to, and its only necessesary with SUM part of the half adder, wich is simply an XOR-gate.

From this our adder will look like this:
\importimagewcaptionw{figures/drawio/Digital/Blokk_ADDER.png}{4-bit added with 8-bit circuit}{0.6}
and our verilog code based on the circuit:
\begin{lstlisting}
module Accumulator ( q ,C ,b );
input [0:7] q ;
input [0:3] C ;
output [0:7] b ;

HALF_ADDER A1 (.A(C[3]) ,.B(q[7]) ,.SUM(b[7]) ,.CARRY(C_in0) );
FULL_ADDER A2 (.C_in(C_in0) , .A(C[2]) , .B(q[6]) ,.SUM(b[6]) ,.C_out(C_in1));	
FULL_ADDER A3 (.C_in(C_in1) , .A(C[1]) , .B(q[5]) ,.SUM(b[5]) ,.C_out(C_in2));	 
FULL_ADDER A4 (.C_in(C_in2) , .A(C[0]) , .B(q[4]) ,.SUM(b[4]) ,.C_out(C_in3));
HALF_ADDER A5 (.A(C_in3) ,.B(q[3]) ,.SUM(b[3]) ,.CARRY(C_in4) );  
HALF_ADDER A6 (.A(C_in4) ,.B(q[2]) ,.SUM(b[2]) ,.CARRY(C_in5) );
HALF_ADDER A7 (.A(C_in5) ,.B(q[1]) ,.SUM(b[1]) ,.CARRY(C_in6) );
xor(b[0], C_in6, q[0]);
endmodule
\end{lstlisting}

Since we know have made arithmetic circuits wich can multipy and add, as well an 8-bit register with the accumulated value in the MAC unit, we can know assemble these circuits. Se figure: 

\importimagewcaptionw{figures/drawio/Digital/MAC.png}{4-bit added with 8-bit circuit}{0.6}
 


\begin{lstlisting}
module MAC ( A ,B ,Y ,CLK1 ,RESET );
input [0:1] A ;
input [0:1] B ;
inout [0:7] Y ;
input CLK1 ;
input RESET ;
wire [0:3] C ;

Multiplier M1( .A(A), .B(B), .C(C) ); 
Accumulator A1( .q(Y) ,.C(C) ,.b(b) ); 
EIGHT_B_REG B1( .b(b) ,.CLK_Out(CLK1) ,.RESET_Out(RESET) ,.q(Y) );
endmodule
\end{lstlisting}

