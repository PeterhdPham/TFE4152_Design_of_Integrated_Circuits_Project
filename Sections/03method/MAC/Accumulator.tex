\subsubsection{Method for Accumulater}
 
The \textbf{D-flip flop with asychron} circuit is based on figure ...ref.. in section.... By assembling eight D-flip flop with eight data inputs $b[0:7]$ and corresponding $q[0:7]$ outputs, we make an \textbf{(8-bit register)}. The registers are connected in parallell by having shared $RESET$ and $CLK$ input. 
!SKRIV MER OM REGISTER-BEGGE SIN DEL SKAL KOMME HER!



The \textbf{Half Adder} and the \textbf{Full Adder} was designed on based on \ref{sec:ADDERS}, see Figure..(figure av Half Adder og Full Adder)

(Få med verilog koden for Half Adder og Full Adder, skriv hva som er LSB og MSB)



Since the output from the 2X2-bits Multiplier is a 4-bit value(denoted as $C$), and the register is a 8 bit-register, with an 8-bit output(reffered as $q$)- we have to assemble half adder and full adder in a series. This arrangement ensures that the $SUM$ output of each adder stage contributes to the updated value of the register. The output from the Multiplier is labeled as $C$, with $C[3]$ assigned as the least significant bit (LSB) and $C[0]$ as the most significant bit (MSB). Conversely, the output from the register, known as $q$, assigns $q[7]$ as the LSB and $q[0]$ as the MSB. In other way, we are using the number scheme MSb 0 [kilde].

For the LSB for both $C$ and $q$ there is abscence of a carry input, therefor only necessesary with a half-adder with two inputs for q[7] and C[3]. For the next three following bits, the requirement for three inputs emerges. Therefore, a full adder is employed for each of the following stages, adding a bit from $C[2]$ to $C[0]$ and the corresponding bits from $q[6]$ to $q[4]$, along with the carry from the previous adder stage. With no more remaining bits from $C$ after the fourth bit, we have absence of an input, therefor it will only be a necessity with the use of a half adders with inputs of the remainig q-bits and carry from the previous adder stage. MSB $q[0]$ does not require a dual-output, as there no no more half adders to transfer a carry bit to, and its only necessesary with SUM part of the half adder, wich is simply an XOR-gate.

 





