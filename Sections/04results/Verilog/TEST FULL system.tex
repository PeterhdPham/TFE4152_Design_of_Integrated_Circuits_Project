\subsection{Test and simulate in Verilog}
To see the results we used verilog as mentioned.
\import{./Verilog}{TEST MAC.tex}
\import{./Verilog}{TEST FSM.tex}
\subsubsection{FULL SYSTEM}
Full wrong- delta delay

\importimagewcaption{figures/Verylog/FULL.wrong.PNG}{WRONG}

\begin{lstlisting}
    module TEST_AWESOME_3;

	reg run;
	reg clk;
	reg reset;
	wire [0:7] c;
	reg [0:1] a;
	reg [0:1] b;
	AWESOME_VOL_3 A3( .A(a) ,.B(b), .C(c) ,.RUN(run) ,.CLK(clk) ,.RESET(reset) );	
	initial begin
			a = 2'b11;
		end	  
	initial begin
		b=2'b01;
	end	
	initial begin
		clk=1'b0;
		forever #1	clk=~clk;
	end
	initial begin
		reset=1'b1;
		#2;
		reset=1'b0;
	end;
	initial begin  
		run=1'b1;
	end;
	initial begin
		#100 $finish;
	end
endmodule
\end{lstlisting}

skriv at det er delta delay- hva vi gjorde legg til bilde av med de ekstra klokkene
\importimagewcaption{figures/Verylog/FULL.right_ideal.PNG}{IDEAL}

\begin{lstlisting}
    module TEST_AWESOME_3;

	reg run;
	reg clk;
	reg reset;
	wire [0:7] c;
	reg [0:1] a;
	reg [0:1] b;
	AWESOME_VOL_3 A3( .A(a) ,.B(b), .C(c) ,.RUN(run) ,.CLK(clk) ,.RESET(reset) );	
	initial begin
			a = 2'b11;
		end	  
	initial begin
		b=2'b01;
	end	
	initial begin
		clk=1'b0;
		forever #1	clk=~clk;
	end
	initial begin
		reset=1'b1; #2; reset=1'b0; #32; reset=1'b1; #10;reset=1'b0;	
	end;
	initial begin  
		run=1'b1;
	end;
	initial begin
		#100 $finish;
	end
endmodule
\end{lstlisting}
\importimagewcaption{figures/Verylog/FULL.RESET.PNG}{RESET}
\begin{lstlisting}
    module TEST_AWESOME_3;

	reg run;
	reg clk;
	reg reset;
	wire [0:7] c;
	reg [0:1] a;
	reg [0:1] b;
	AWESOME_VOL_3 A3( .A(a) ,.B(b), .C(c) ,.RUN(run) ,.CLK(clk) ,.RESET(reset) );	
	initial begin
			a = 2'b11;
		end	  
	initial begin
		b=2'b01;
	end	
	initial begin
		clk=1'b0;
		forever #1	clk=~clk;
	end
	initial begin
		reset=1'b1;
		#2;
		reset=1'b0;
	end;
	initial begin  
		run=1'b1; #32; run=1'b0; #10; run=1'b1;
	end;
	initial begin
		#100 $finish;
	end
endmodule
\end{lstlisting}
\importimagewcaption{figures/Verylog/FULL.RUN.PNG}{RUN}
