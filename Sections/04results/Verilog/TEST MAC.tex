\subsubsection{Result MAC}
For testing the MAC-unit, we tested the subcirciuts, multipliers, adder, and 8-bit register seperarly.

For testing the \textbf{multiplier} we tested it with all possible combination of two 2-bit product, as shown in the testbench.. 
\begin{lstlisting}
    `timescale 1 ns / 1 ps
module TESTMULTIPLIER;
	reg [0:1] a;  
	reg [0:1] b;
	wire [0:3] c;  
	Multiplier testmultiplikator( .A(a), .B(b), .C(c) );	// Test all combinations
  	initial begin
		  a[1]=1'b0; 
		  forever #5 a[1]=~a[1];
	  end
	  initial begin
		  a[0]=1'b0; 
		  forever #10 a[0]=~a[0];
		  end
	initial begin
		  b[1]=1'b0; 
		  forever #20 b[1]=~b[1];
	end 
	initial begin
		  b[0]=1'b0; 
		  forever #40 b[0]=~b[0];
	  end
	initial begin
		#80 $finish; // End the simulation
	end
endmodule
\end{lstlisting}
\importimagewcaption{figures/Verylog/Simulation_multiplier.PNG}{Simulation of the multiplier}

The simulations showed excepted product output($C$) value, of all $A$ and $B$ inputs combination, showed in Figure \ref{fig:figures/Verylog/Simulation_multiplier.PNG}.

For testing the \textbf{8-bit register} we set the input($b[0:7]$) to a random 8-bit value for every 12$ns$. The input was set to $RESET_{OUT}=0$ in the middle of the inputsignal $CLK_{OUT}$ - this way we could ceritfy that it is in fact asychron reset. 

\begin{lstlisting}
module TESTEIGHTBREG;
	reg [0:7] B;
	reg clk_out;
	reg reset_out; 
	wire [0:7] Q ;	
	EIGHT_B_REG EIGHTB( .b(B) ,.CLK_Out(clk_out) ,.RESET_Out(reset_out) ,.q(Q) ); 
	initial begin
		clk_out=1'b0;
		forever #10	clk_out=~clk_out;
	end	  
	initial begin
		reset_out =1'b0;  #15; reset_out =1'b1; #20; reset_out =1'b0; #5; reset_out=1'b1;
	end	 
	initial begin
  		forever	 #12 B = $random; //random 8-bit value as input
	end	
	initial begin
		#100 $finish;
	end	  
endmodule	
\end{lstlisting}

\importimagewcaption{figures/Verylog/Simulation_register.PNG}{Simulation of the 8-bits register}
The result for the 8-bit register, illustrated in Figure  \ref{fig:figures/Verylog/Simulation_register.PNG} happened as excepted: The $RESET_{OUT}$ inputsignal, were in fact asychron as it reset the output in the middle of $CLK_{OUT}$ signal, marked with a red circle on the Figure \ref{fig:figures/Verylog/Simulation_register.PNG}. The 8-bit register got also only updated in the rising clock edge of $CLK_{OUT}$, offcours when $RESET_{OUT}=1$. 


For testing the \textbf{Adding} part, we inteded to test two extreme scenario where both input, $q$ and $C$ were zero, and when $q$=255 and $C$=1, by testing this we would see what would happend in the begining and in the end (if the MAC-unit output gets up to 255 in the 8-bit register). Afterward we set in random inputs for - to see if it behaved as it should do for any random number, this was updated in every 10ns, as shown in the testbench.

\begin{lstlisting}
module TEST_ACCUMULATOR;
	reg [0:7] Q;
	reg [0:3] c;
	wire [0:7] B;  
	Accumulator TESTACCUMULATOR( .q(Q) ,.C(c) ,.b(B) );//random inputs	  
	initial begin  //extreme scenario
		Q=255; c=1;
		#5
		Q=0; c=0;
		end
	initial begin 
  		forever	 #10 Q = $random;
	end	  
	initial begin
  		forever	 #10 c = $random;
	end	
	initial begin
		#100 $finish;
	end	  
endmodule	
\end{lstlisting}
\importimagewcaption{figures/Verylog/Simulation_adder.PNG}{Simulation of the adder}
The output, illustrated in Figure \ref{fig:figures/Verylog/Simulation_MAC.PNG} as excepted. Testing the extremal values both had both an output 00, this is excepted since we are not considering overflow in the project, refferering to 255+1. The other values also had excepted output from the two inputs. 

For testing the \textbf{MAC}, we used random inputs and tested the asychron reset.
\begin{lstlisting}
    `timescale 1 ns / 1 ps
module TEST_MAC;
	reg [0:1] a;
	reg [0:1] b;
	reg clk;
	reg reset;
	wire [0:7] y;
	MAC TESTMAC ( .A(a) ,.B(b) ,.Y(y) ,.CLK1(clk) ,.RESET(reset) );	
	initial begin
		reset=1'b0; #10; reset=1'b1; #22; reset=1'b0; #10; reset=1'b1;	
	end
		initial begin
			a = 2'b01;
  		forever	 #10 a = $random;
        end	  
	initial begin
		b=2'b01;
  		forever	 #10 b = $random;
	end	
	initial begin
		clk=1'b0;
		forever #5	clk=~clk;
	end	
	initial begin
		#100 $finish;
	end	
endmodule	
\end{lstlisting}
\importimagewcaption{figures/Verylog/Simulation_MAC.PNG}{Simulation of the MAC}


