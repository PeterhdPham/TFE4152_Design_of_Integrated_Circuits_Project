\section{Introductions}

Artificial Intelligence (AI) is changing many parts of our lives, and the computational requirements for AI algorithms have surged immensely.  At the heart of these computations lies the Multiply Accumulate (MAC) operation, a fundamental process at the chip level for AI \cite{wang_2021_an}. These MAC circuits are also used in other areas, like signal processing \cite{ntnudepartmentofelectronicsystems_2023_project}.

A lot of researchers are trying to make these MAC circuits work better. One idea is called in-memory computing. This means doing the MAC operations right where the data is stored, so we don't waste time moving it around, this can be implemented using multi-bit storage capability, long cycling endurance, simple erase/write operation, etc \cite{wang_2021_an}. There are different types of memory we can use, but each one has its own problems. One big problem is that these circuits use power even when they're not doing anything, which is called static (leakage) power consumption \cite{ntnudepartmentofelectronicsystems_2023_project}.

Static power consumption, an inherent byproduct of semiconductor devices, has profound implications for overall power usage, system reliability, and operational costs. As the document "TFE4152 - Digital Design and Computer Architecture - Project" underscores, the design of a MAC unit that judiciously manages static power consumption is of paramount importance. Furthermore, the design intricacies extend beyond mere component selection, encompassing circuit topology, transistor dimensions, and supply voltage considerations, among others \cite{ntnudepartmentofelectronicsystems_2023_project}. The intricacies associated with the design, especially concerning power consumption, necessitate a thoughtful exploration of various strategies and technologies.

Recent advancements, such as those highlighted by Zhang et al. (2021), present a pioneering MAC design that merges multiply-add operations, potentially addressing some of the power consumption challenges \cite{zhang_2021_a}. However, the path to an optimized MAC unit remains rife with challenges.

Given this backdrop, our research project seeks to delve into the design and optimization of a MAC unit with a keen focus on minimizing static power consumption. By building on the foundational studies and addressing the identified gaps, this project aims to contribute significantly to the ongoing efforts of creating energy-efficient hardware tailored for AI and other computational applications.




Reducing this static power usage is important. It's not just about picking the right parts but also about how they're put together and how they're set up \cite{ntnudepartmentofelectronicsystems_2023_project}. Some new designs, like the one from Zhang et al. (2021), are trying different ways to solve this problem \cite{zhang_2021_a}. But there's still a lot we don't know.

So, our project is looking into how to design a MAC circuit that uses less static power. We're building on what others have found out, and we hope to find ways to make these circuits better for AI and other uses.