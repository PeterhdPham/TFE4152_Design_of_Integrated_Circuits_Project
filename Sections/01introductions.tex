\section{Introductions}

Artificial Intelligence (AI) has undeniably carved its niche in contemporary discussions, both among the general populace and experts in the field. Just last year, the introduction of ChatGPT in November 2022 reignited debates surrounding the potential and boundaries of AI. Amidst the ongoing deliberations about the correct and ethical use of AI, one thing remains clear: AI is not just a fleeting trend but a transformative force that's here to stay.

In the realm of electronics and integrated circuits, the rise of AI presents both challenges and opportunities. At the heart of AI computations, particularly in neural networks, lies a fundamental operation: the Multiply-Accumulate (MAC). A plethora of AI tasks require a series of multiplications followed by additions, making MAC units an indispensable component of AI hardware. The speed and efficiency of these units directly influence the performance of AI applications. Hence, designing a robust and efficient MAC unit is paramount to harnessing the full potential of AI.

However, while the significance of MAC units in AI hardware is well acknowledged, there exists a gap in designing units that are both efficient and adaptable to the evolving needs of AI applications. Addressing this void, the present research delves into the design of a MAC unit, aiming to contribute a solution that not only meets the current demands but also anticipates future computational challenges.\cite{ntnudepartmentofelectronicsystems_2023_project}