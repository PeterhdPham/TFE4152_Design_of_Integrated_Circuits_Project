\section{Verilog VHDL}


We need to assemble the circuits: the combinational circuits in the FSM, and the arithmetic circuits within the MAC unit. Additionally, we must implement an 8-bit register. Using Verilog, we'll ensure that all components operate in sync as intended.

\subsection{Half-adderer}
\importimagewcaption{Verilog-Half-adder.png}{}
\importimagewcaptionw{Half-adderer.png}{}{1}
\begin{table}[h]
    \centering
    \begin{tabular}{|c|c|c|c|}
    \hline
    \(A\) & \(B\) & \(S\) & \(C\) \\
    \hline
    0 & 0 & 0 & 0 \\
    0 & 1 & 1 & 0 \\
    1 & 0 & 1 & 0 \\
    1 & 1 & 0 & 1 \\
    \hline
\end{tabular}
\caption{Truthtable for Half Adder}                 
\end{table}


\subsection{Adderer}
\importimagewcaption{Verilog-Adder.png}{}
\importimagewcaptionw{Test-Adder.png}{}{1}
\begin{table}[h]
    \centering
    \begin{tabular}{|c|c|c|c|c|}
    \hline
    \(A\) & \(B\) & \(C_{\text{in}}\) & \(S\) & \(C_{\text{out}}\) \\
    \hline
    0 & 0 & 0 & 0 & 0 \\
    0 & 1 & 0 & 1 & 0 \\
    1 & 0 & 0 & 1 & 0 \\
    1 & 1 & 0 & 0 & 1 \\
    0 & 0 & 1 & 1 & 0 \\
    0 & 1 & 1 & 0 & 1 \\
    1 & 0 & 1 & 0 & 1 \\
    1 & 1 & 1 & 1 & 1 \\
    \hline
    \end{tabular}
    \caption{Sannhetstabell for Full Adder}
    \end{table}
    
    
    \subsection{Mulitplikator}
    \importimagewcaption{Verilog-Multiplikator.png}{}
    \importimagewcaption{Test-Mulitplikator.png}{}
\begin{table}[h]
    \centering
    \begin{tabular}{|c|c|c|c|c|c|c|c|}
    \hline
    \(A_1\) & \(A_0\) & \(B_1\) & \(B_0\) & \(C_3\) & \(C_2\) & \(C_1\) & \(C_0\) \\
    \hline
    0 & 0 & 0 & 0 & 0 & 0 & 0 & 0 \\
    0 & 0 & 0 & 1 & 0 & 0 & 0 & 0 \\
    0 & 0 & 1 & 0 & 0 & 0 & 0 & 0 \\
    0 & 0 & 1 & 1 & 0 & 0 & 0 & 0 \\
    0 & 1 & 0 & 0 & 0 & 0 & 0 & 0 \\
    0 & 1 & 0 & 1 & 0 & 0 & 0 & 1 \\
    0 & 1 & 1 & 0 & 0 & 0 & 1 & 0 \\
    0 & 1 & 1 & 1 & 0 & 0 & 1 & 1 \\
    
    1 & 0 & 0 & 0 & 0 & 0 & 0 & 0 \\
    1 & 0 & 0 & 1 & 0 & 0 & 1 & 0 \\
    1 & 0 & 1 & 0 & 0 & 1 & 0 & 0 \\
    1 & 0 & 1 & 1 & 0 & 1 & 1 & 0 \\
    1 & 1 & 0 & 0 & 0 & 0 & 1 & 0 \\
    1 & 1 & 0 & 1 & 0 & 1 & 0 & 1 \\
    1 & 1 & 1 & 0 & 0 & 1 & 1 & 0 \\
    1 & 1 & 1 & 1 & 1 & 0 & 0 & 1 \\
    \hline
    \end{tabular}
    \caption{Sannhetstabell for 2-bit multiplikator}
    \end{table}


\subsection{D-flip flop with asychron reset}
    \importimagewcaption{Verilog-D-flipflop.png}{}
    \importimagewcaption{D-flipflop.png}{}
    \centering
        \begin{tabular}{|c|c|c|c|c|}
        \hline
        \(D\) & \text{CLK (edge)} & \(R\) & \(Q \text{ (next)}\) & \(\overline{Q} \text{ (next)}\) \\
        \hline
        x & \(\downarrow\) or no change & 1 & \(Q \text{ (prev)}\) & \(\overline{Q} \text{ (prev)}\) \\
        x & \(\uparrow\) & 1 & \(D\) & \(\overline{D}\) \\
        x & x & 0 & 0 & 1 \\
        \hline
        \end{tabular}
 